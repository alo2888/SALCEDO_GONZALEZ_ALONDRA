\documentclass[12pt,letterpaper]{article}


\usepackage[T1]{fontenc}
\usepackage[utf8]{inputenc}
\usepackage{times}
\usepackage[left=3.5cm,top=2.5cm,bottom=2.5mm,right=3.5cm]{geometry}

\begin{document}

\maketitle

\title{Primer avance: Estacionamiento Giratorio horizontal.\\ 
Universidad Politécnica de la Zona Metropolitana de Guadalajara.\\ 
Ing Mecatrónica. Maestro: Ing. Carlos Enrique Morán Garabito.}\\
\author{Nombres: Capuchino González Jonathan Alejandro, Fernández Gaeta Uriel y Salcedo González Alondra}

\section{Planteamiento del problema}
Solucionar la problemática de estacionamiento en zona residencial usando un mecanismo en la cual permite al usuario a guardar su carro de manera segura y de fácil manejo de este.

\section{Objetivos general del proyecto}
Elaborar un estacionamiento cuyas funciones sean fáciles y útiles al usuario para mayor comodidad.\\
Diseñar mecanismos para larga duracion y de fácil matenimiento.

\section{Objetivo del proyecto}
Brindar seguridad tanto al usuario como al coche que se está guardando en el.

\section{Justifcación}
Innovar los estacionamientos inteligentes con la manera horizontal de este.

\section{Bibliografía}
-Carrousel [Archivo de video]. (2016, 15 abril). Recuperado 20 de agosto, 2019, de https://www.youtube.com/watch?v=0w9yCadwD00\\
-Carrusel. Mecanismo del motor [Archivo de video]. (2016, 29 de octubre). Recuperado 20 de agosto, 2019, de https://www.youtube.com/watch?v=A7iFozSOWZI\\
-MOTOR Siemens 1LeO T 180 HP 6 P - Sea Ingeniería [Archivo de video]. (s.f). Recuperado 20 de agosto, 2019, de https://seaing.cl/motores-electricos/1144-motor-siemens-1leO-t-180-hp-6-p.html\\
-Elevador Estacionamiento [Archivo de video]. (2010, 27 de marzo). Recuperado 20 de agosto, 2019 de https://www.youtube.com/watch?v=D370WEJ95RA\\
-Rotary Parking System [Archivo video]. (2012, 23 noviembre). Recuperado 20 de agosto, 2019, de https://www.youtube.com/watch?v=XC48qZ2TIZM

\end{document}